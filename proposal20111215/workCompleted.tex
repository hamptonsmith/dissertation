%----------------------------------------------------------------------------
\section{Work Completed}\label{sect:workCompleted}
%----------------------------------------------------------------------------
In this document, we have proposed to design a new mathematical subsystem for a mechanical verifier, experiment with specification best-practices, and create a minimalist prover and library of components to evaluate these developments.  The design of this system will be based on our experience writing and verifying reusable components using the existing RESOLVE verifying compiler, as well as applying lessons from our previous research.

The current RESOLVE design is already a workable prototype containing many of our ideas about flexible, robust mathematical systems and generic, reusable specifications.  As we have previously published, we have applied organizational principles from modern programming languages to the design of our mathematical system\cite{SmithSAVCBS2008}---ideas like modular development and the separation of signature from realization.  In the cited paper, we highlighted how these techniques improve the reusability of mathematical results and encourage the amortization of effort invested in deep mathematical theorems as opposed to the relatively more simple VCs arising from code.

In addition, in order to more fully test our hypothesis regarding the straightforwardness of VCs, we have shifted the RESOLVE compiler away from it's previous proof backend, Isabelle\cite{Isabelle}, to a new, homebrew minimalistic prover\cite{SmithICSRResolve} designed as a platform for prover experimentation.  Armed with this, and in conjunction with our sister group at The Ohio State University, we have published some of our initial findings on the complexity of VCs arrising from practical, well-engineered programs\cite{deepMathematics}, finding that the majority of VCs arising from a such programs were very shallow and often tautological, requiring only one or two proof steps.

With a prover against which to measure our progress, and a prototype mathematical and specification system to experiment on, we have begun to explore the engineering considerations inherent in choosing a correct specification, which influences both the modular design of programs intended for verification, but may also provide insight into the language features most useful in a practical verifier.  We have published about this initial experimentation, the library we developed to experiment on, and some of the alternative metrics we have proposed for measuring the difficulty of a VC in \cite{SmithSAVCBS2010}, where we demonstrate that the chosen mathematical model for a component can have a large impact on the difficulty of resulting VCs.  We further discuss the development of reusable components in our library, particularly with respect to their usefulness outside of RESOLVE, in \cite{ICSRJava}.  Building on all of this, in \cite{MapChallengeProblem} we discuss a common component, Map, and highlight multiple dimensions of desirable attributes for a Map specification which might be permitted by a flexible verification system, including verifiability, genericity, and others.
