


\chapter{Evaluation of Minimalist Prover\label{ch:proverEvaluation}}
%-----------------------------------------------------------------------------
The evaluation of our minimalist prover orbits three fundamental questions: first, does our prover succeed in verifying programs; second, are our heuristics effective at increasing the usefulness of the prover; and third, does the data gathered by our prover expose any interesting properties of the VCs that arise from well-engineered programs.

The first and second question are explored by considering a series of verification benchmarks.  In Section \ref{canProve} we present each benchmark, then explore the effectiveness of the automated prover when applied to that benchmark.  Following that, in Section \ref{evalHeuristics}, we use those same benchmarks to explore how our various heuristics impact the effectiveness of the prover.  Finally, the third question is addressed in Section \ref{proverEvalConclusion}, where we provide some observations and conclusions based on the collected data.

This chapter focuses on the specifications and realizations of the benchmarks, with discussion of relevant mathematics forming the basis of Chapter \ref{ch:mathEvaluation}.  We will generally present only enough of each specification and realization to motivate our discussion, but the full specifications of each benchmark and any associated datastructures can be found in Appendix \ref{apx:spec} and full realizations in Appendix \ref{apx:real}.  Full listings of the proofs generated as solutions to the benchmarks in this chapter are available in Appendix \ref{apx:proofs}.

While the usability of our prover and its application in educational settings is not a core part of this thesis, we take this opportunity to note that our minimalistic prover has provided the backbone of our educational web-interface for three years now, and point the interested reader at \cite{something}, \cite{something}, and \cite{something} for examples of research where our prover is applied in educational settings.


%-----------------------------------------------------------------------------
\section{Description of Metrics}
%-----------------------------------------------------------------------------
For this chapter we focus on four primary metrics for exploring the effectiveness of the prover and the difficulty of the VCs.  These are:

\begin{itemize}
	\item \textbf{VCs proved.}  The number of VCs that were successfully dispatched by the prover.  Since the proof space is quite large, the prover often runs with a set timeout, after which it gives up, in which case this metric refers to the number of VCs that were proved within the timeout.  Where relevant, the timeout will be stated.
	\item \textbf{Real time.}  The amount of time required to dispatch the VC, generally measured in milliseconds.  Because this metric can vary depending on a number of variables, we will generally use the median of five repeated trials.
	\item \textbf{Proof steps.}  The number of relevant steps taken by the prover.  One of the features of the prover is to prune steps that did not impact the final result.
	\item \textbf{Search steps.}  A subset of the overall proof steps including only those steps taken during the ``consequent exploration'' phase described in Section \ref{consequentExploration}, and not including those consequent steps that are ``obvious'', namely: replacing a consequent that matches a given with \texttt{true}, replacing a consequent that is a symmetric equality with \texttt{true}, and eliminating a conjunct that is \texttt{true}.  Intuitively this metric refers to the number of steps that were blind exploration of the proof space and which may have to be backtracked over, rather than deterministic steps over which we do not backtrack.
\end{itemize}


%-----------------------------------------------------------------------------
\section{Benchmark Solutions\label{canProve}}
%-----------------------------------------------------------------------------

%-----------------------------------------------------------------------------
	\subsection{VSTTE Benchmarks}
%-----------------------------------------------------------------------------
In 2010, members of the RSRG at Clemson and The Ohio State University published a set of incremental verification benchmarks at the Verified Software: Tools, Theories, and Experiments workshopVSTTE\cite{benchmarks}.  These benchmarks were intended to provide a basis for experimentation and discussion between verification efforts.  In \cite{heatherDissertation}, the author presents a selection of these benchmarks implemented in RESOLVE in order to demonstrate the VC-generation capabilities of the RESOLVE compiler.  We mirror that methodology in this section, presenting a selection of the VSTTE benchmarks, including several that are borrowed or adapted from \cite{heatherDissertation}, before applying our minimalist prover to them and presenting resulting data.
	%-----------------------------------------------------------------------------
		\subsubsection{Benchmark 1: Adding and Multiplying Integers}	%-----------------------------------------------------------------------------

\paragraph{Problem Statement}Verify an operation that adds two numbers by repeated incrementing. Verify an operation that multiplies two numbers by repeated addition, using the first operation to do the addition. Make one algorithm iterative, the other recursive.\footnote{This and all other problem statements quote directly from \cite{benchmarks}.}

\paragraph{Solution Discussion}In Listing \ref{lst:integerAddMultSpec} we present specifications of integer addition and multiplication operations in RESOLVE.  Each of these operations is specified to work ``in place'', transforming the first parameter (which is also used as one of the inputs) into the final solution.

\lstinputlisting[float=h,language=resolve,caption={The specification of \texttt{Adding\_Capabilitiy} and \texttt{Multiplying\_Capability}\label{lst:integerAddMultSpec}}]{proverEval/examples/IntegerPlusTimesSpecs.en}

Each specifies that its second parameter must be positive---dealing correctly with the presence of negative numbers adds a surprising amount of implementation complexity for a language like RESOLVE that strives to correctly reason about bounded integers, since the maximum and minimum integer are generally not symmetrical.  Note, however, that we lose no generality---a more general addition or multiplication function could be specified and wrapped around the ones presented in this section, transforming any addition or multiplication task into a finite subset of tasks that could computed correctly by these implementations.

We experimented both with an iterative and recursive implementation of \texttt{Adding\_Capability}.  We present the rescursive implementation in Listing \ref{lst:addImpl}.  The iterative implementation is available in Appendix \ref{apx:spec}.

\lstinputlisting[float=h,language=resolve,caption={A recursive implementation of \texttt{Adding\_Capabilitiy}\label{lst:addImpl}}]{proverEval/examples/Recursive_Add_to_Realiz.rb}

Our implementation of \texttt{Multiplying\_Capabilitiy} is iterative and presented in Listing \ref{lst:multImpl}.

\lstinputlisting[float=h,language=resolve,caption={An iterative implementation of \texttt{Multiplying\_Capabilitiy}\label{lst:multImpl}}]{proverEval/examples/Iterative_Multiply_into_Realiz.rb}

\begin{sloppypar}
Integers are partially built-in to RESOLVE, which makes using one enhancement to \texttt{Integer\_Template} from another fairly unweildy, which is why we do not use \texttt{Adding\_Capability} to implement \texttt{Multiplying\_Capability} as requested by the benchmark, but rather the normal integer \texttt{+}.  We note, however, that \texttt{+} is drawn from \texttt{Integer\_Template} as a normal specification and thus presents the same verification challenges (that is: reasoning about it is not built in).
\end{sloppypar}

\paragraph{Results}Each of our three implementations is fully verifiable.  The results of those verifications, with associated metrics is presented in Figure \ref{fig:addMultResults}.

\begin{figure}
	\centering
	\begin{subfigure}[b]{0.4\textwidth}
		\centering
		\begin{tabular}{lrrr}
			\toprule
				& Time (ms)	& Steps	& Search \\
			\midrule
			VC 0\_1	& 5418		& 5	& 0     \\
			VC 0\_2	& 3388		& 5	& 0     \\
			VC 0\_3	& 520		& 10	& 0     \\
			VC 0\_4	& 754		& 9	& 0     \\
			VC 0\_5	& 3841		& 6	& 0     \\
			VC 0\_6	& 616		& 8	& 0     \\
			VC 0\_7	& 528		& 5	& 0     \\
			VC 1\_1	& 4846		& 9	& 0     \\
			\bottomrule
		\end{tabular}
		\caption{Iterative \texttt{Adding\_Capabilitiy} results\label{fig:iterAddResults}}
	\end{subfigure}
	\qquad
	\begin{subfigure}[b]{0.4\textwidth}
		\centering
		\begin{tabular}{lrrr}
			\toprule
				& Time (ms)	& Steps	& Search \\
			\midrule
			VC 0\_1	& 2494		& 9	& 0     \\
			VC 0\_2	& 1225		& 9	& 0     \\
			VC 0\_3	& 931		& 5	& 0     \\
			VC 0\_4	& 1448		& 6	& 0     \\
			VC 0\_5	& 1086		& 6	& 0     \\
			VC 0\_6	& 745		& 8	& 0     \\
			VC 0\_7	& 1157		& 5	& 0     \\
			VC 1\_1	& 334		& 8	& 0     \\
			\bottomrule
		\end{tabular}
		\caption{Recursive \texttt{Adding\_Capabilitiy} results\label{fig:recAddResults}}
	\end{subfigure}

	\vspace{2em}
	\begin{subfigure}[b]{0.6\textwidth}
		\centering
		\begin{tabular}{lrrr}
			\toprule
				& Time (ms)	& Steps	& Search \\
			\midrule
			VC 0\_1	& 6201		& 7	& 0     \\
			VC 0\_2	& 3513		& 5	& 0     \\
			VC 0\_3	& 3535		& 7	& 2     \\
			VC 0\_4	& 342		& 7	& 0     \\
			VC 0\_5	& 307		& 5	& 0     \\
			VC 1\_1	& 5854		& 8	& 0     \\
			\bottomrule
		\end{tabular}
		\caption{Iterative \texttt{Multiplying\_Capabilitiy} results\label{fig:recAddResults}}
	\end{subfigure}
  \caption{Results from verification of Benchmark 1 solutions}
\end{figure}


%-----------------------------------------------------------------------------
		\subsubsection{Benchmark 2: Benchmark 2: Binary Search an Array}	%-----------------------------------------------------------------------------

\paragraph{Problem Statement}Verify an operation that uses binary search to find a given entry in an array of entries that are in sorted order.

\paragraph{Solution Discussion}In Listing \ref{lst:searchSpec} we present a specification of a searching operation on an array.  The operation takes as its input an entry and an array in sorted order, then uses a parameterizable comparison function, \texttt{LEQ}, to search the array.

\lstinputlisting[float=h,language=resolve,caption={The specification of \texttt{Searching\_Capabilitiy}\label{lst:searchSpec}}]{proverEval/examples/Search_Capability.en}

The \emph{requires} clause of the operation uses higher-order definitions to establish that the array is in order---i.e., that it is \emph{conformal} with the given comparator.  The \emph{ensures} clause states that the operation will return \texttt{true} if and only if the given entry exists between the lower and upper bound of the array.  The details of these higher-order definitions are explored more fully in Chapter \ref{ch:mathEvaluation}.

An implementation for this operation is provided in Listing \ref{lst:searchImpl}.

\lstinputlisting[float=h,language=resolve,caption={An implementation of \texttt{Searching\_Capabilitiy}\label{lst:searchImpl}}]{proverEval/examples/Bin_Search_Realiz.rb}

The realization takes a function, \texttt{Are\_Ordered()} which provides a programmatic way of establishing \texttt{LEQ}.  From there, the implementation is the usual straightforward binary search implementation.  Note that when calculating the new mid, we first take the difference, then divide, avoiding the overflow problem exposed in \cite{thingy}.

While we provide syntactic sugar for array operations, they are supported by an ordinary component that provides specifications for all array operations, a snippet of which is provided in Listing \ref{lst:arraySpec}.

\lstinputlisting[float=h,language=resolve,caption={A snippet of the specification of arrays\label{lst:arraySpec}}]{proverEval/examples/Static_Array_Template.co}

\paragraph{Results} \textbf{XXX fill this section in when we have results from binary searh XXX}

\FloatBarrier
%-----------------------------------------------------------------------------
		\subsubsection{Benchmark 3: Sorting a Queue}	%-----------------------------------------------------------------------------

\paragraph{Problem Statement}Specify a user-defined FIFO queue ADT that is generic (i.e., parameterized by the type of entries in a queue). Verify an operation that uses this component to sort the entries in a queue into some client-defined order.

\paragraph{Solution Discussion}In Listing \ref{lst:sortingSpec} we present a specification of a queue sorting enhancement.  As with Benchmark 2, it takes as a parameter a client-provided definition to define the sorted order, \texttt{LEQV}.  It uses two higher-order definitions \texttt{Is\_Coformal\_With()} and \texttt{Is\_Permutation()} to ensure that the final queue is ordered according to \texttt{LEQV} and contains the same elements as those original provided, respectively.

\lstinputlisting[float=h,language=resolve,caption={The specification of \texttt{Sorting\_Capabilitiy}\label{lst:sortingSpec}}]{proverEval/examples/Sorting_Capability.en}

A straightforward selection sort realization is provided in Listing \ref{lst:sortingImpl}.

\lstinputlisting[float=h,language=resolve,caption={A selection sort realization of \texttt{Sorting\_Capabilitiy}\label{lst:sortingImpl}}]{proverEval/examples/Selection_Sort_Realization.rb}

Again, as in Benchmark 2, we take a client-provided operation, \texttt{Compare()}, that implements \texttt{LEQV}, and use it to implement our \texttt{Remove\_Min()} sub-operation.  The \texttt{One()} and \texttt{Two()} procedures simply provide a way of getting the programmatic values \texttt{1} and \texttt{2} using a normal specification.

\paragraph{Results}This implementation is fully verifiable.  The results of that verifications, with associated metrics, is presented in Figure \ref{fig:sortingResults}.

\begin{figure}
	\centering
	\begin{tabular}{lrrr}
		\toprule
			& Time (ms)	& Steps	& Search \\
		\midrule
		VC 0\_1	& 1179		& 6 	& 0     \\
		VC 0\_2	& 445		& 5 	& 0     \\
		VC 0\_3	& 878		& 5 	& 0     \\
		VC 0\_4	& 1298		& 6 	& 1     \\
		VC 0\_5	& 2789		& 8	& 0     \\
		VC 0\_6	& 2947		& 9 	& 3     \\
		VC 0\_7	& 1438		& 8 	& 0     \\
		VC 0\_8	& 1685		& 10	& 1     \\
		VC 0\_9	& 2220		& 9 	& 3     \\
		VC 1\_1	& 516 		& 5 	& 0     \\
		VC 1\_2	& 912 		& 10	& 1     \\
		VC 2\_1	& 731 		& 5 	& 0     \\
		VC 2\_2	& 1816		& 8 	& 0     \\
		VC 2\_3	& 1708		& 5 	& 0     \\
		VC 2\_4	& 1557		& 6 	& 1     \\
		VC 2\_5	& 2826		& 14	& 0     \\
		\bottomrule
	\end{tabular}
	\qquad
	\begin{tabular}{lrrr}
		\toprule
			& Time (ms)	& Steps	& Search \\
		\midrule
		VC 2\_6	& 4360		& 12	& 3     \\
		VC 2\_7	& 1864		& 10	& 2     \\
		VC 2\_8	& 1869		& 7 	& 0     \\
		VC 3\_1	& 795 		& 5 	& 0     \\
		VC 3\_2	& 1834		& 8 	& 0     \\
		VC 3\_3	& 1657		& 5 	& 0     \\
		VC 3\_4	& 1473		& 6 	& 1     \\
		VC 3\_5	& 2842		& 14	& 0     \\
		VC 3\_6	& 4027		& 10	& 1     \\
		VC 3\_7	& 1755		& 10	& 2     \\
		VC 3\_8	& 1851		& 7 	& 0     \\
		VC 4\_1	& 759 		& 5 	& 0     \\
		VC 4\_2	& 2858		& 12	& 0     \\
		VC 4\_3	& 1154		& 5 	& 0     \\
		VC 4\_4	& 2864		& 18	& 3     \\
		\bottomrule
	\end{tabular}
	\caption{Selection sort \texttt{Sorting\_Capabilitiy} results\label{fig:sortingResults}}
\end{figure}

\FloatBarrier
%-----------------------------------------------------------------------------
	\subsection{Other Benchmarks}
%-----------------------------------------------------------------------------

%-----------------------------------------------------------------------------
		\subsubsection{Benchmark 4: Reversing a Queue}	%-----------------------------------------------------------------------------

\paragraph{Problem Statement}Specify a user-defined FIFO queue ADT that is generic (i.e., parameterized by the type of entries in a queue). Verify an operation that reverses the entries in a given queue.

\paragraph{Solution Discussion}In Listing \ref{lst:flippingSpec} we present a specification of a queue reversing enhancement called \texttt{Flipping\_Capability}.  A recursive solution follows in Listing \ref{lst:flippingImpl}.


\lstinputlisting[float=h,language=resolve,caption={The specification of \texttt{Flipping\_Capabilitiy}\label{lst:flippingSpec}}]{proverEval/examples/Flipping_Capability.en}

\lstinputlisting[float=h,language=resolve,caption={A recursive realization of \texttt{Flipping\_Capabilitiy}\label{lst:flippingImpl}}]{proverEval/examples/Recursive_Flipping_Realiz.rb}

\paragraph{Results}This implementation is fully verifiable.  The results of that verifications, with associated metrics, is presented in Figure \ref{fig:flippingResults}.

\begin{figure}
	\centering
	\begin{tabular}{lrrr}
		\toprule
			& Time (ms)	& Steps	& Search \\
		\midrule
		VC 0\_1	& 1441		& 5 	& 0     \\
		VC 0\_2	& 3055		& 7 	& 0     \\
		VC 0\_3	& 2679		& 8 	& 0     \\
		VC 0\_4	& 2127		& 9 	& 2     \\
		VC 1\_1	& 335		& 10	& 0     \\
		\bottomrule
	\end{tabular}
	\caption{Recursive \texttt{Flipping\_Capabilitiy} results\label{fig:flippingResults}}
\end{figure}


%-----------------------------------------------------------------------------
		\subsubsection{Benchmark 5: Array Realization of Stack}	%-----------------------------------------------------------------------------

\paragraph{Problem Statement}Specify a user-defined LIFO stack ADT that is generic (i.e., parameterized by the type of entries in a queue). Verify an implementation of that array.

\paragraph{Solution Discussion}Listing \ref{lst:stackSpec} gives RESOLVE's specification for a stack in the context of a \texttt{Stack\_Template}, which is generic and bounded.

\lstinputlisting[float=h,language=resolve,caption={The specification of \texttt{Stack}\label{lst:stackSpec}}]{proverEval/examples/Stack_Template.co}

We implement this specification on top of an array, which, as in Benchmark 2, is a first-class component with its own template and operation specifications.  This array-based implementation is provided in Listing \ref{lst:stackImpl}.

\lstinputlisting[float=h,language=resolve,caption={An array-based implementation of \texttt{Stack}\label{lst:flippingImpl}}]{proverEval/examples/Array_Realiz.rb}

\paragraph{Results}This implementation is fully verifiable.  The results of that verifications, with associated metrics, is presented in Figure \ref{fig:stackResults}.

\begin{figure}
	\centering
	\begin{tabular}{lrrr}
		\toprule
			& Time (ms)	& Steps	& Search \\
		\midrule
		VC 0\_1	& 1441		& 5 	& 0     \\
		VC 1\_1	& 3055		& 7 	& 0     \\
		VC 2\_1	& 2679		& 8 	& 0     \\
		VC 2\_2	& 2127		& 9 	& 2     \\
		VC 2\_3	& 335		& 10	& 0     \\
		VC 3\_1	& 795 		& 5 	& 0     \\
		VC 3\_2	& 1834		& 8 	& 0     \\
		VC 3\_3	& 1657		& 5 	& 0     \\
		VC 3\_4	& 1473		& 6 	& 1     \\
		VC 3\_5	& 2842		& 14	& 0     \\
		VC 4\_1	& 795 		& 5 	& 0     \\
		VC 4\_2	& 1834		& 8 	& 0     \\
		VC 4\_3	& 1657		& 5 	& 0     \\
		VC 4\_4	& 1473		& 6 	& 1     \\
		\bottomrule
	\end{tabular}
	\qquad
	\begin{tabular}{lrrr}
		\toprule
			& Time (ms)	& Steps	& Search \\
		\midrule
		VC 4\_5	& 2842		& 14	& 0     \\
		VC 5\_1	& 795 		& 5 	& 0     \\
		VC 5\_2	& 1834		& 8 	& 0     \\
		VC 5\_3	& 1657		& 5 	& 0     \\
		VC 5\_4	& 1473		& 6 	& 1     \\
		VC 6\_1	& 795 		& 5 	& 0     \\
		VC 6\_2	& 1834		& 8 	& 0     \\
		VC 6\_3	& 1657		& 5 	& 0     \\
		VC 6\_4	& 1473		& 6 	& 1     \\
		VC 7\_1	& 795 		& 5 	& 0     \\
		VC 7\_2	& 1834		& 8 	& 0     \\
		VC 7\_3	& 1657		& 5 	& 0     \\
		VC 7\_4	& 1473		& 6 	& 1     \\
		\bottomrule
	\end{tabular}
	\caption{Array-based \texttt{Stack} implementation results\label{fig:stackResults}}
\end{figure}
