%
% thesis.tex
%
% Master's Thesis/Ph.D. Dissertation Template
% Clemson University
%

%
% The document guidelines say the font can be between 10pt and 12pt.
% Specify whatever you want it to be here.
%
\documentclass[10pt]{ClemsonThesis}

% Packages and macros
\inputfile{setup.tex}


%
% Make the document your own -- fill in these values to reflect the type of
% document you are writing.
%
\title{Engineering Specifications and Mathematics\\for Verified Software}
\department{School of Computing}
\documentType{Dissertation}
\major{Computer Science}
\degree{Doctor of Philosophy}
\graduationMonth{May}
\graduationYear{2013}
\author{Hampton Smith}
\committeeChair{Dr. Murali Sitaraman}
\committeeMemberOne{Dr. Brian C. Dean}
\committeeMemberTwo{Dr. Jason O. Hallstrom}
\committeeMemberThree{Dr. Roy P. Pargas}
%% optional (for Master's) \committeeMemberThree{Dr. John Doe}
%% optional \committeeMemberFour{Dr. Jane Doe}
%% optional \committeeMemberFive{Dr. Mary Doe}
%% optional \committeeMemberSix{Dr. Mark Doe}

%
% PDF Setup -- most of this you do not need to touch
%
\hypersetup{
    colorlinks,
    linkcolor={black},
    citecolor={black},
    filecolor={black},
    urlcolor={black},
    pdftitle={\theTitle},
    pdfauthor={\theAuthor},
    pdfsubject={\theDocumentType},
    pdfkeywords={Clemson University, \theDepartment, \theDocumentType, \theMajor, \theDegree},
    pdfstartpage={1},
}


%
% User-specified command definitions/redefinitions
%
%% \newcommand{\cplusplus}{{\rm C\raise.5ex\hbox{\small ++}}}
%% \newcommand{\num}[1]{\mbox{(\textit{#1})}}
%% \renewcommand{\ttdefault}{pcr}
%% \renewcommand\lstlistlistingname{List of Listings}


\begin{document}
%  ============================================================================
    \frontmatter % Begin front matter (pages are numbered with Roman numerals)
%  ============================================================================

    \addtotoc{Title Page}{\maketitle}          % Generate the title page
    \doublespacing                             % Text should be double spaced
    \setcounter{page}{2}                       % Abstract begins on page 2
    \addtotoc{Abstract}{\begin{abstract}
At the heart of the argument for formal, mathematical methods of software quality assurance is that increased energy spent to develop formal specifications and prove software components against those specifications is amortized over the lifetime of the verified component.  Thus, modularity and reuse are central prerequisites of practical verification.  Because of this, there are two strategies for reducing effective energy invested in a verified component: 1) decrease the amount of effort required to verify it, and 2) increase its reusability and thus its lifetime.

While many modern verification systems exist, few seem to have been designed with modularity and reuse in mind.  On the one hand are systems built on industrial, object-oriented languages, which provide modularity in the programming world, but whose specifications rely on mathematics that do not support these goals.  On the other hand are extensible, generic mathematical systems that are not integrated with programming languages that support component reuse.  The result, on both sides, is the creation of components insufficiently generic and extensible to be reused.

As we wish to verify components of increasing complexity, we must build these components out of smaller subcomponents.  If these subcomponents display poor modularity, they will compose poorly or not at all, resulting in complex interactions and proof obligations that are difficult to satisfy.  To date, modern systems have addressed this complexity by placing the onus of verification on a suite of sophisticated, industrial-strength automated theorem provers that are on the bleeding edge of artificial intelligence design.

This seems paradoxical, however, as programmers do not often rely on deep mathematical results when they reason about the correctness of components.  We posit that by shifting the burden to the design of good components and specifications, as supported by a flexible mathematical and specification subsystem, proof obligations should become much more obvious and more easily-proved.  In addition to influencing the design of our own system, RESOLVE, such exploration would benefit other systems as well, by informing specification and theory design across the board.

The intent of this proposal is three-fold.  First, to develop a flexible mathematical framework for program specification designed with modularity and reuse in mind.  Second, to experiment with the design and implementation of a minimalist prover sufficiently flexible to operate on that mathematical framework and determine the actual practical requirements for program verification of well-specified components.  Third, to develop a diverse library of components specified using a variety of specification styles in order to identify best-practices in specification engineering by measuring the complexity of resultant VCs.
\end{abstract}
}  % Generate the abstract

    %
    % The dedication page is optional.  Comment out this line if you do not
    % want to include this page.
    %
    \addtotoc{Dedication}{\chapter*{Dedication}
For David and Thomas, who got me through this alive.
}

    %
    % The acknowledgment page is optional.  Comment out this line if you do
    % not want to include this page.
    %
    \addtotoc{Acknowledgments}{\chapter*{Acknowledgments}
This research was supported by the National Science Foundation, grants CCF-0811748, CCF-1161916, and DUE-1022941.  We would like to acknowledge members of the RSRG here at Clemson and in our sister group at OSU.  In particular, we would like to thank Bill Ogden for providing the original motivation for many of our mathematical features and Bruce Weide for his support and frequent feedback.  We would also like to acknowledge the members of the research committee for their assistance focussing and supporting this research: Brian C. Dean, Jason O. Hallstrom, Roy P. Pargas, and Murali Sitaraman.

On a personal note, I would like to acknowledge my parents, Wade and Delores Smith, for their relentless and continuing support, without which this dissertation would not have been possible.
}

    \singlespacing                             % Single space the lists
    \tableofcontents \clearpage                % Generate the Table of Contents

    %
    % REMEMBER: Review your caption listings in the genrated lists
    %           and make sure they include '\newline' commands as necessary.
    %           See the README for further information.
    %
    \addtotoc{List of Tables}{\listoftables}   % Generate the List of Tables
    \addtotoc{List of Figures}{\listoffigures} % Generate the List of Figures
    \addtotoc{List of Listings}{\lstlistoflistings}



%  ===========================================================================
    \mainmatter % Begin main matter (pages are numbered with Arabic numerals)
%  ===========================================================================
    \doublespacing % Text should be double spaced

    %
    % Here we have each chapter in a separate file.  Name these as you choose,
    % and include them in the order you want them to appear.  Be sure to use
    % the \inputfile command.
    %
    \inputfile{introduction}
    \inputfile{background}
    \inputfile{resolvebackground}
    \inputfile{mathSystem}
    \inputfile{prover}
    \inputfile{evaluation}
    \inputfile{conclusion}

    %
    % The appendices are optional.  This is the format for two or more.
    % If you do not wish to include an appendix, comment out these lines.
    % If you want just one, see the formatting guidelines.
    %
    \begin{appendices}
        \begin{subappendices}
            \inputfile{appendixA.tex}
            \inputfile{appendixB.tex}
            \inputfile{appendixC.tex}
        \end{subappendices}
    \end{appendices}



    \singlespacing                             % Single space the Bibliography

    %
    % The bibliography style.  Set this to whatever matches you discipline.
    % For example, Computer Science would likely use 'plain'.  You might
    % also want to change the name from 'Bibliography' to 'References'
    % or 'Work Cited'.
    %
    % 'plain'   gets you numbered references and citations (e.g., [1] Dyson).
    %
    % 'alpha'   gets you labels formed from an abbreviation of the authors'
    %           names and the year of publication.  If there is more than
    %           one author, it will use the first letter of up to the first
    %           three authors' last names.
    %
    %           Some examples:
    %               [DED01] F.W. Dyson, A.G. Edgar, and D.B. Denny ... 2001
    %               [DE01] F.W. Dyson, A.G. Edgar ... 2001
    %               [Dys01] F.W. Dyson ... 2001
    %
    % 'apalike' gets you labels formed from the authors' names and year of
    %           publication.
    %
    %           Some examples:
    %               [Dyson et al., 2001] F.W. Dyson, A.G. Edgar, and
    %                 D.B. Denny ... 2001
    %               [Dyson and Edgar, 2001] F.W. Dyson, A.G. Edgar ... 2001
    %               [Dyson, 2001] F.W. Dyson ... 2001
    %
    \bibliographystyle{plain}
    \addtotoc{Bibliography}{\bibliography{bibliography}}
\end{document}
