%-----------------------------------------------------------------------------
\chapter{Evaluation}\label{sec:evaluation}
%-----------------------------------------------------------------------------
In some sense, this research is itself an evaluation suite---a prover is an apparatus for collecting data about provability and a library of components specified in different ways along well-defined axes of variability is a set of good test cases.  However, in order to get good data from our test suite, we must make sure it is an effective one.

%-----------------------------------------------------------------------------
\section{Extensible, Flexible Mathematics for Specification}\label{sec:evalMath}
%-----------------------------------------------------------------------------
Since the motivation for the design of the mathematical and specification subsystem is specifically to enable the development of modular components specified using techniques from a flexible set of mathematical options, the creation of the component library in the third part of the problem statement will serve to evaluate the system.  It will be evaluated by the handful of programming and mathematical professionals here at Clemson, at Ohio State, and elsewhere, who will contribute to this library.  The system will be a success if components can be easily created with the modularity and genericity profiles we desire and require improvement if they cannot.  Additionally, a number of benchmarks discussed in the related works section (Section \ref{sec:overviewEngineering}) will be useful in demonstrating that we are able to specify components in reasonable ways.

%-----------------------------------------------------------------------------
\section{Minimalist Prover}\label{sec:evalProver}
%-----------------------------------------------------------------------------
Our minimalist prover must be evaluated to be certain it is a prover worth exploring: after all, if it cannot prove any practical components, it can't collect useful data for us.  Between the RESOLVE groups at OSU and here at Clemson, we already have a broad library of specified components on which to experiment and, as stated in Section \ref{sec:researchProver}, the prover has already been used to excellent effect in a number of deployments.  In addition, the library we develop for the third part of our problem statement will provide an ideal set of components against which to evaluate if our prover can be reasonably called upon to verify practical components.  As we discussed in the research section, a constellation of components that cannot be mechanically verified by our prover may indicate a missing feature to be implemented.

The final set of features for our prover will not be decided arbitrarily, but rather via quantitative data on the gains over a number of verifiability metrics (discussed further in the next section) under different feature configurations.  This will permit us to experimentally arrive at a prover that strikes a good balance between simplicity and power.

In addition, in evaluating the components in the library, we will expect the prover to be able to provide data on a number of proof metrics, which will require further prover developments.  As we have many existing components, both mechanically verifiable and not, we will have a strong indicator of how accurate and useful our metrics are before ever applying them to the unknowns of our new library.

%-----------------------------------------------------------------------------
\section{Specification and Mathematical Engineering}\label{sec:evalEngineering}
%-----------------------------------------------------------------------------
This is the meat of our evaluation, which the other two subproblems support.  We would like to be able to draw conclusions about techniques for mathematical and specification development.  With a diverse library of components developed, each specified in multiple different ways along well-defined axes of variability, we will be able to evaluate these techniques by applying our minimalist prover to gather verifiability metrics.  In addition to time-to-verify, which is the standard metric used in the literature, in \cite{smithSpecificationAbstractions} we discuss two other potential metrics: required proof length and theorem specificity.  Other metrics may present themselves as we explore and be folded into our prover\footnote{In that same paper we noted the usefulness of metrics that could be obtained even for proofs that could not be completed.  We are on the lookout for these.}.  These metrics will be collected for each component (including the most important metric of all---can it be mechanically verified?), then analyzed to draw some conclusion about the nature of specification.
